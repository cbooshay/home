% preamble

\documentclass[12pt]{article}

% preamble

% TO DOUBLESPACE THE PRINTOUT, INSERT THE COMMAND
% \renewcommand{\baselinestretch}{2}

%\setlength{\textheight}{8.5in}
%\setlength{\textwidth}{6.25in}
%\setlength{\topmargin}{0.0in}

\newtheorem{defn}{Definition}
\newtheorem{cor}[defn]{Corollary}
\newtheorem{lemma}[defn]{Lemma}
\newtheorem{obs}[defn]{Observation}
\newtheorem{prop}[defn]{Proposition}
\newtheorem{thm}[defn]{Theorem}
\newtheorem{cond}[defn]{Condition}
\newtheorem{conj}[defn]{Conjecture}
\newtheorem{ass}[defn]{Assumption}
\newtheorem{example}[defn]{Example}
\newtheorem{rem}[defn]{Remark}

\newcommand{\abs}[1]{\left| #1 \right| }
\newcommand{\ans}{\noi\textbf{Answer: }}
\newcommand{\ds}{\displaystyle}
\newcommand{\dydx}{\ds \frac{dy}{dx}}
\newcommand{\infnorm}[1]{\ensuremath{\left\| #1 \right\|_{\infty}}}
\newcommand{\ital}{\textit}
\newcommand{\la}{\langle}
\newcommand{\lb}{\left\{}
\newcommand{\li}{\mathrm{Li}}
\newcommand{\limn}{\lim_{n\rightarrow\infty}}
\newcommand{\lp}{\left(}
\newcommand{\Mod}[1]{\ (\mathrm{mod}\ #1)}
\newcommand{\N}{I\!\!N}
\newcommand{\noi}{\noindent}
\newcommand{\norm}[1]{\ensuremath{\left\| #1 \right\| }}
\newcommand{\oon}{\frac{1}{n}}
\newcommand{\pic}[1]{\begin{center}\includegraphics{#1}\end{center}}
\newcommand{\R}{I\!\!R}
\newcommand{\ra}{\rangle}
\newcommand{\rb}{\right\}}
\newcommand{\rp}{\right)}
\newcommand{\skp}{\vspace{\baselineskip}}
\newcommand{\snsp}{@!@!@!@!@!}
\newcommand{\trm}{\textrm}
\newcommand{\ve}{\ensuremath{\varepsilon}}

% document

\usepackage{amsfonts}
\usepackage{amsmath}
\usepackage{amssymb}
\usepackage{amsthm}
\usepackage{color}
\usepackage{float}
\usepackage{graphicx}
\usepackage{hyperref}
\usepackage{times}
\usepackage{url}
\usepackage{xcolor}

% document

\begin{document}

\section*{Formulas from Legendre and Kummer \begin{small} \begin{color}{gray} \\
(\today) \end{color} \end{small}}

If the base $p$ digits of $n$ are $\ds d_0, d_1, \ldots, d_k$, and $\ds d_i \leq (p-1)/2$ for all $i$, then the base $p$ digits of $2n$ are $\ds 2d_0, 2d_1, \ldots, 2d_k$, which means that $\ds S_p(2n) = 2 S_p(n)$. 

 and so $\ds v_p(\gamma_n)=0$. For example, $\ds 27=220_5$, so it should be the case that $\ds v_5(\gamma_{27}) =0$ (and it is). As $\ds v_p(\gamma_n)=0$ if and only if $\ds S_p(2n) = 2 S_p(n)$, the primes that do not appear in the prime factorization of $\ds \gamma_n$ are precisely those primes $p$ such that none of the base $p$ digits of $n$ is more than $\ds (p-1)/2$.

Let $q = (p-1)/2$ for a prime $p>3$, and let $\ds \lb \gamma_{n_k} \rb$ be the subsequence of $\ds \lb \gamma_n \rb$ such that $\ds v_p(\gamma_{n_k})=0$. $\ds n_k$ is the number whose base $p$ digits are the base $q$ digits of $k$. For example, suppose we want to find the $\ds 147^{th}$ positive integer $\alpha$ such that $\ds v_{13}(\gamma_{\alpha})=0$. First we express $\ds 147$ in base $\ds (13-1)/2=6$: $\ds 147 = 403_6$. Then $\ds \alpha$ is the number with this base 13 representation, so 
\[
  \alpha = 4\times 13^2 + 0\times 13 + 3\times 13 = 679
\]
and one can check that $\ds v_{13}(\gamma_{679}) = 0$.

Conjecture: $\ds v_p(\gamma_n) = 1$ if and only if the set of base $p$ digits of $n$ has exactly one element that is larger than $\ds (p-1)/2$.


\iffalse
A few observations that will hopefully lead to a proof of this useful theorem. First,  
\[
  v_p(\gamma_n) = v_p( (2n)!) - 2 v_p(n!)
\]

This implies that $\ds v_p\lp \gamma_n \rp = \frac{2 S_p(n) - S_p(2n)}{p-1}$. $\ds S_2(2n) = S_2(n)$, and if $n$ is a power of 2, then $\ds S_2(n)=1$ and so $\ds v_2 \lp \gamma_n \rp = 1$.

A consequence of this, as $\ds v_p(\gamma_n)$ must be a nonnegative integer, is that for any prime $p$, $\ds 2 S_p(n) - S_p(2n)$ must be divisible by $\ds p-1$.





\section*{Erd\"{o}s Conjecture}

Erd\"{o}s conjectured that that for $\ds n>4$, $\ds \gamma_n$ is not squarefree, which is equivalent to saying that for $n>4$, there is some prime $p$ such that $\ds p^2$ divides $\ds \gamma_n$. As we have seen, any such prime must be fairly small, i.e., must come from the range $\ds \lb 1,2,\ldots, 2n/3\rb$. What's more, a consequence of the observation that $\ds v_2(\gamma_n) = S_2(n)$ made in Section \ref{KummersTheorem} is that if $\ds S_2(n)>1$, then $\ds \gamma_n$ is divisible by four and so not square free. The only positive integers $n$ with $\ds S_2(n)\leq 1$ are powers of two, so if there is a positive integer $\ds n$ such that $\ds \gamma_n$ is not squarefree, then $n$ must be a power of two.

For a positive integer $n$, $n$ \textbf{primorial} is the product of the primes less than or equal to $n$ and is expressed $\ds n\# $. So
\begin{eqnarray*}
 10\# & = & (2)(3)(5)(7) = 210 \\
 15 \# & = & (2)(3)(5)(7)(11)(13) = 30030
\end{eqnarray*}
The prime factorization of $\ds \gamma_n$ includes one copy of every prime in the range $\ds \lb n+1, n+2, \ldots, 2n\rb$. The product of such primes is $\ds \frac{(2n)\#}{n \#}$. It can be shown that $\ds \frac{(2n)\#}{n \#} \leq 3.439^n$.



The number on the right of (\ref{FactoringGamma50}) is the product of ten consecutive primes, from the $\ds 16^{th}$ prime through the $\ds 25^{th}$ prime. 
So the number in (\ref{FactoringGamma50}) is $\ds \frac{97 \#}{52 \#}$.



\begin{thebibliography}{99} 

\bibitem{HANSON} Hanson, Denis. ``On the Product of the Primes." Canadian Mathematical Bulletin, volume 15, issue 1, March, 1972, pp. 33-37.

\end{thebibliography}
\fi


\end{document}

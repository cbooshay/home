% preamble

\documentclass[12pt]{article}

% preamble

% TO DOUBLESPACE THE PRINTOUT, INSERT THE COMMAND
% \renewcommand{\baselinestretch}{2}

%\setlength{\textheight}{8.5in}
%\setlength{\textwidth}{6.25in}
%\setlength{\topmargin}{0.0in}

\newtheorem{defn}{Definition}
\newtheorem{cor}[defn]{Corollary}
\newtheorem{lemma}[defn]{Lemma}
\newtheorem{obs}[defn]{Observation}
\newtheorem{prop}[defn]{Proposition}
\newtheorem{thm}[defn]{Theorem}
\newtheorem{cond}[defn]{Condition}
\newtheorem{conj}[defn]{Conjecture}
\newtheorem{ass}[defn]{Assumption}
\newtheorem{example}[defn]{Example}
\newtheorem{rem}[defn]{Remark}

\newcommand{\abs}[1]{\left| #1 \right| }
\newcommand{\ans}{\noi\textbf{Answer: }}
\newcommand{\ds}{\displaystyle}
\newcommand{\dydx}{\ds \frac{dy}{dx}}
\newcommand{\infnorm}[1]{\ensuremath{\left\| #1 \right\|_{\infty}}}
\newcommand{\ital}{\textit}
\newcommand{\la}{\langle}
\newcommand{\lb}{\left\{}
\newcommand{\lp}{\left(}
\newcommand{\N}{I\!\!N}
\newcommand{\noi}{\noindent}
\newcommand{\norm}[1]{\ensuremath{\left\| #1 \right\| }}
\newcommand{\oon}{\frac{1}{n}}
\newcommand{\pic}[1]{\begin{center}\includegraphics{#1}\end{center}}
\newcommand{\R}{I\!\!R}
\newcommand{\ra}{\rangle}
\newcommand{\rb}{\right\}}
\newcommand{\rp}{\right)}
\newcommand{\skp}{\vspace{\baselineskip}}
\newcommand{\snsp}{@!@!@!@!@!}
\newcommand{\trm}{\textrm}
\newcommand{\ve}{\ensuremath{\varepsilon}}

% document

\usepackage{amsmath}
\usepackage{graphicx} 
\usepackage{hyperref}
\usepackage{soul}
\usepackage{xcolor}

% document

\begin{document}

\section*{Wordle and the Wolfram Language \begin{small} \begin{color}{gray} \\
(\today) \end{color} \end{small}}

The Wolfram Language's built in \verb# DictionaryLookup[] # function allows for some optimization of strategy at the game of Wordle, in which players attempt to guess a five-letter word, being informed after each guess what letters the hidden word has in common with their guess and whether those letters are in the correct position. Running

\begin{verbatim}
Select[DictionaryLookup[], StringLength[#] == 5 &] // Length
\end{verbatim}

\noi reveals that there are $6789$ five-letter English words (at least in the Wolfram Language's dictionary). We can determine the number of occurrences of each letter of the alphabet in these words via

\begin{verbatim}
Reverse[
 SortBy[
  Tally[
   Flatten[
    Characters /@ 
     Select[DictionaryLookup[], StringLength[#] == 5 &]]], 
  Last]
 ]\end{verbatim}
 
This reports that the five most common letters are, in decreasing order of frequency, ``e", ``s", ``a", ``r", and ``o," which in turn suggests that the best opening word would be a five-letter word spelled with those letters. Being a better coder than a speller, I ran 

\begin{verbatim}
Intersection[
 StringJoin /@ Permutations[{"e", "s", "a", "r", "o"}], 
 Select[DictionaryLookup[], StringLength[#] == 5 &]
 ]
 \end{verbatim}

\noi to discover that the only such word is "arose."

\end{document}

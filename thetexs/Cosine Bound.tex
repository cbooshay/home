% preamble

\documentclass[12pt]{article}

% preamble

% TO DOUBLESPACE THE PRINTOUT, INSERT THE COMMAND
% \renewcommand{\baselinestretch}{2}

%\setlength{\textheight}{8.5in}
%\setlength{\textwidth}{6.25in}
%\setlength{\topmargin}{0.0in}

\newtheorem{defn}{Definition}
\newtheorem{cor}[defn]{Corollary}
\newtheorem{lemma}[defn]{Lemma}
\newtheorem{obs}[defn]{Observation}
\newtheorem{prop}[defn]{Proposition}
\newtheorem{thm}[defn]{Theorem}
\newtheorem{cond}[defn]{Condition}
\newtheorem{conj}[defn]{Conjecture}
\newtheorem{ass}[defn]{Assumption}
\newtheorem{example}[defn]{Example}
\newtheorem{rem}[defn]{Remark}

\newcommand{\abs}[1]{\left| #1 \right| }
\newcommand{\ans}{\noi\textbf{Answer: }}
\newcommand{\ds}{\displaystyle}
\newcommand{\dydx}{\ds \frac{dy}{dx}}
\newcommand{\infnorm}[1]{\ensuremath{\left\| #1 \right\|_{\infty}}}
\newcommand{\ital}{\textit}
\newcommand{\la}{\langle}
\newcommand{\lb}{\left\{}
\newcommand{\lp}{\left(}
\newcommand{\N}{I\!\!N}
\newcommand{\noi}{\noindent}
\newcommand{\norm}[1]{\ensuremath{\left\| #1 \right\| }}
\newcommand{\oon}{\frac{1}{n}}
\newcommand{\pic}[1]{\begin{center}\includegraphics{#1}\end{center}}
\newcommand{\R}{I\!\!R}
\newcommand{\ra}{\rangle}
\newcommand{\rb}{\right\}}
\newcommand{\rp}{\right)}
\newcommand{\skp}{\vspace{\baselineskip}}
\newcommand{\snsp}{@!@!@!@!@!}
\newcommand{\trm}{\textrm}
\newcommand{\ve}{\ensuremath{\varepsilon}}

% document

\usepackage{amsmath}
\usepackage{graphicx} 
\usepackage{hyperref}
\usepackage{soul}
\usepackage{xcolor}

% document

\begin{document}

\section*{Cosine Bound \begin{small} \begin{color}{gray} (January 2, 2025) \end{color} \end{small}}

The cosine function is a good lower bound for the standard normal density function near zero.

\pic{images/CosineAndZ}

To establish the bound, let $\ds g(x) = \log \left[\frac{e^{-x^2/2}}{\cos(x)}\right]$, so $\ds g'(x) = \tan(x) - x$. From Figure \ref{TanxMinusX}, the area of $\ds \triangle OAB$ is larger than that of the circular sector $\ds OAC$, so 
\[
  \frac{\tan(x)}{2} \geq \frac{x}{2} 
\]
Thus, for $\ds 0\leq x\leq \pi/2$, $\ds g'(x)\geq 0$ and $g(0)=0$, so $g(x)\geq 0$. Because $g$ is an even function, this means $\ds g(x) \geq 0$ for $\ds -\pi/2 \leq x\leq \pi/2$, which is equivalent to $\ds \cos(x) \leq e^{-x^2/2}$.

\begin{figure}[H]
\centering
\caption{$\ds \tan(x) \geq x$}
\pic{images/TanxMinusX}
\label{TanxMinusX}
\end{figure}

This suggests that a good lower bound for textbook standard normal probabilities within one standard deviation of the mean is
\begin{equation} \label{SineApprox}
 P(0\leq Z\leq a) \geq \int_0^a \cos(x)\, dx = \sin(a)
\end{equation}
Figure \ref{SinApproxError} shows the difference between both sides of \ref{SineApprox} for $\ds 0\leq a \leq 1$.

\begin{figure}[H]
\centering
\caption{$\ds P(0\leq Z\leq a) - \sin(a)$}
\pic{images/SinApproxError}
\label{SinApproxError}
\end{figure}


\end{document}

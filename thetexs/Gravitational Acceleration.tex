% preamble

\documentclass[12pt]{article}

% preamble

% TO DOUBLESPACE THE PRINTOUT, INSERT THE COMMAND
% \renewcommand{\baselinestretch}{2}

%\setlength{\textheight}{8.5in}
%\setlength{\textwidth}{6.25in}
%\setlength{\topmargin}{0.0in}

\newtheorem{defn}{Definition}
\newtheorem{cor}[defn]{Corollary}
\newtheorem{lemma}[defn]{Lemma}
\newtheorem{obs}[defn]{Observation}
\newtheorem{prop}[defn]{Proposition}
\newtheorem{thm}[defn]{Theorem}
\newtheorem{cond}[defn]{Condition}
\newtheorem{conj}[defn]{Conjecture}
\newtheorem{ass}[defn]{Assumption}
\newtheorem{example}[defn]{Example}
\newtheorem{rem}[defn]{Remark}

\newcommand{\abs}[1]{\left| #1 \right| }
\newcommand{\ans}{\noi\textbf{Answer: }}
\newcommand{\ds}{\displaystyle}
\newcommand{\dydx}{\ds \frac{dy}{dx}}
\newcommand{\infnorm}[1]{\ensuremath{\left\| #1 \right\|_{\infty}}}
\newcommand{\ital}{\textit}
\newcommand{\la}{\langle}
\newcommand{\lb}{\left\{}
\newcommand{\li}{\mathrm{Li}}
\newcommand{\limn}{\lim_{n\rightarrow\infty}}
\newcommand{\lp}{\left(}
\newcommand{\Mod}[1]{\ (\mathrm{mod}\ #1)}
\newcommand{\N}{I\!\!N}
\newcommand{\noi}{\noindent}
\newcommand{\norm}[1]{\ensuremath{\left\| #1 \right\| }}
\newcommand{\oon}{\frac{1}{n}}
\newcommand{\pic}[1]{\begin{center}\includegraphics{#1}\end{center}}
\newcommand{\R}{I\!\!R}
\newcommand{\ra}{\rangle}
\newcommand{\rb}{\right\}}
\newcommand{\rp}{\right)}
\newcommand{\skp}{\vspace{\baselineskip}}
\newcommand{\snsp}{@!@!@!@!@!}
\newcommand{\trm}{\textrm}
\newcommand{\ve}{\ensuremath{\varepsilon}}

% document

\usepackage{amsfonts}
\usepackage{amsmath}
\usepackage{amssymb}
\usepackage{amsthm}
\usepackage{color}
\usepackage{float}
\usepackage{graphicx}
\usepackage{hyperref}
\usepackage{times}
\usepackage{url}
\usepackage{xcolor}

% document

\begin{document}

\section*{Gravitational Acceleration \begin{small} \begin{color}{gray} \\
(November 4, 2025) \end{color} \end{small}}

Newton's Law of Gravitation says that the gravitational force between an object of mass $\ds m_1$ and an object of mass $m_2$ whose centers are at a distance $r$ is 
\[
  F = \frac{G m_1 m_2}{r^2}
\]
where 
\[
  G = 6.6742 \times 10^{-11} m^3 s^{-2} kg^{-1}
\]
If one of the objects is the Earth, then $\ds m_1 \approx 5.972 \times 10^{24}$ kg. Suppose the other is some small (relative to the mass of the Earth) mass $\ds m_2$ on the surface of the Earth. Then $\ds r \approx 6.378 \times 10^6$ m, and 
\begin{align*}
  F &= m_2 \lp \frac{\lp 6.6742 \times 10^{-11} \textup{m}^3 \textup{s}^{-2} \textup{kg}^{-1} \rp \lp 5.972 \times 10^{24} \textup{kg} \rp}{\lp 6.378 \times 10^6 \textup{m} \rp^2} \rp \\
  &= m_2 \lp 9.798 \, \textup{m}\, \textup{s}^{-2} \rp
\end{align*}
which gives the familiar constant of gravitational acceleration. 

If a mass is at a height $z$ above the surface of the Earth, so $\ds r = r_E + z$, then $\ds F = mg$, where 
\begin{align*}
  g &= \frac{G m_E}{(r_E+z)^2} \\
  &= \frac{G m_E}{r_E^2} \cdot \frac{r_E^2}{(r_E+z)^2} \\
  &= g_0 \lp \frac{r_E^2}{(r_E+z)^2} \rp
\end{align*}
If $\ds \alpha$ is the ratio of $z$ to the radius of the Earth, the right hand side becomes $\ds g_0 \lp \frac{1}{(1+\alpha)^2}\rp$, and we see how $g$ shrinks with $\ds \alpha$.

The top of Mount Everest is about $\ds 8849$ meters above the surface of the Earth, giving an $\ds \alpha-$value of $\ds .00139$, and $\ds g = .997 g_0$. To lower the gravitational constant to 90\% of its surface-of-the-Earth value, one would have to be about 4\% of an Earth radius (roughly 214 miles) above the surface.


\end{document}

% preamble

\documentclass[12pt]{article}

% preamble

% TO DOUBLESPACE THE PRINTOUT, INSERT THE COMMAND
% \renewcommand{\baselinestretch}{2}

%\setlength{\textheight}{8.5in}
%\setlength{\textwidth}{6.25in}
%\setlength{\topmargin}{0.0in}

\newtheorem{defn}{Definition}
\newtheorem{cor}[defn]{Corollary}
\newtheorem{lemma}[defn]{Lemma}
\newtheorem{obs}[defn]{Observation}
\newtheorem{prop}[defn]{Proposition}
\newtheorem{thm}[defn]{Theorem}
\newtheorem{cond}[defn]{Condition}
\newtheorem{conj}[defn]{Conjecture}
\newtheorem{ass}[defn]{Assumption}
\newtheorem{example}[defn]{Example}
\newtheorem{rem}[defn]{Remark}

\newcommand{\abs}[1]{\left| #1 \right| }
\newcommand{\ans}{\noi\textbf{Answer: }}
\newcommand{\ds}{\displaystyle}
\newcommand{\dydx}{\ds \frac{dy}{dx}}
\newcommand{\infnorm}[1]{\ensuremath{\left\| #1 \right\|_{\infty}}}
\newcommand{\ital}{\textit}
\newcommand{\la}{\langle}
\newcommand{\lb}{\left\{}
\newcommand{\li}{\mathrm{Li}}
\newcommand{\limn}{\lim_{n\rightarrow\infty}}
\newcommand{\lp}{\left(}
\newcommand{\Mod}[1]{\ (\mathrm{mod}\ #1)}
\newcommand{\N}{I\!\!N}
\newcommand{\noi}{\noindent}
\newcommand{\norm}[1]{\ensuremath{\left\| #1 \right\| }}
\newcommand{\oon}{\frac{1}{n}}
\newcommand{\pic}[1]{\begin{center}\includegraphics{#1}\end{center}}
\newcommand{\R}{I\!\!R}
\newcommand{\ra}{\rangle}
\newcommand{\rb}{\right\}}
\newcommand{\rp}{\right)}
\newcommand{\skp}{\vspace{\baselineskip}}
\newcommand{\snsp}{@!@!@!@!@!}
\newcommand{\trm}{\textrm}
\newcommand{\ve}{\ensuremath{\varepsilon}}

% document

\usepackage{amsfonts}
\usepackage{amsmath}
\usepackage{amssymb}
\usepackage{amsthm}
\usepackage{color}
\usepackage{float}
\usepackage{graphicx}
\usepackage{hyperref}
\usepackage{times}
\usepackage{url}
\usepackage{xcolor}

% document

\begin{document}

\section*{More Applications of Kummer \begin{small} \begin{color}{gray} (\today) \end{color} \end{small}}

Kummer's Theorem gives the formula (\ref{VpOfGamman2}) relating $\ds v_p(\gamma_n)$ to $\ds S_p(n)$ and $\ds S_p(2n)$, where $\ds S_p(x)$ is the sum of the base $p$ digits of $x$. This can be leveraged to give some remarkable conclusions about prime factorizations of very large numbers based on the digits of a much smaller one.

\begin{equation} \label{VpOfGamman2}
  v_p(\gamma_n) = \frac{2 S_p(n) - S_p(2n)}{p-1}
\end{equation}

Doubling a number with small digits like $\ds 312$ is easy because we can just double each digit in its place: $\ds 2\times 312 = 624$. Doubling a number like $\ds 473$ is not so easy because in doubling the tens place, we wind up with ``too many" tens. More generally and precisely, if the base $p$ digits of $n$ are $\ds d_0, d_1, \ldots, d_k$, and $\ds d_i \leq (p-1)/2$ for all $i$, then the base $p$ digits of $2n$ are $\ds 2d_0, 2d_1, \ldots, 2d_k$, which means that $\ds S_p(2n) = 2 S_p(n)$ and so $\ds v_p(\gamma_n)=0$. For example, $\ds 27=220_5$; none of the base 5 digits of 27 is more than $\ds (5-1)/2$; so it should be the case that $\ds v_5(\gamma_{27}) =0$ (and it is). 

In essence, if the base $p$ representation of $n$ contains all ``small" digits, then $p$ does not appear in the prime factorization of $\ds \gamma_n$. If $\ds n = 3220102_7$, then $\ds \gamma_n$ has nearly a quarter million digits, but no appearance of seven in its prime factorization.

\end{document}

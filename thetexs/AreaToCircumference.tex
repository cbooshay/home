% \documentclass[12pt]{article}

\documentclass[12pt]{article}

% preamble

% TO DOUBLESPACE THE PRINTOUT, INSERT THE COMMAND
% \renewcommand{\baselinestretch}{2}

%\setlength{\textheight}{8.5in}
%\setlength{\textwidth}{6.25in}
%\setlength{\topmargin}{0.0in}

\newtheorem{defn}{Definition}
\newtheorem{cor}[defn]{Corollary}
\newtheorem{lemma}[defn]{Lemma}
\newtheorem{obs}[defn]{Observation}
\newtheorem{prop}[defn]{Proposition}
\newtheorem{thm}[defn]{Theorem}
\newtheorem{cond}[defn]{Condition}
\newtheorem{conj}[defn]{Conjecture}
\newtheorem{ass}[defn]{Assumption}
\newtheorem{example}[defn]{Example}
\newtheorem{rem}[defn]{Remark}

\newcommand{\abs}[1]{\left| #1 \right| }
\newcommand{\ans}{\noi\textbf{Answer: }}
\newcommand{\ds}{\displaystyle}
\newcommand{\dydx}{\ds \frac{dy}{dx}}
\newcommand{\infnorm}[1]{\ensuremath{\left\| #1 \right\|_{\infty}}}
\newcommand{\ital}{\textit}
\newcommand{\la}{\langle}
\newcommand{\lb}{\left\{}
\newcommand{\li}{\mathrm{Li}}
\newcommand{\limn}{\lim_{n\rightarrow\infty}}
\newcommand{\lp}{\left(}
\newcommand{\Mod}[1]{\ (\mathrm{mod}\ #1)}
\newcommand{\N}{I\!\!N}
\newcommand{\noi}{\noindent}
\newcommand{\norm}[1]{\ensuremath{\left\| #1 \right\| }}
\newcommand{\oon}{\frac{1}{n}}
\newcommand{\pic}[1]{\begin{center}\includegraphics{#1}\end{center}}
\newcommand{\R}{I\!\!R}
\newcommand{\ra}{\rangle}
\newcommand{\rb}{\right\}}
\newcommand{\rp}{\right)}
\newcommand{\skp}{\vspace{\baselineskip}}
\newcommand{\snsp}{@!@!@!@!@!}
\newcommand{\trm}{\textrm}
\newcommand{\ve}{\ensuremath{\varepsilon}}

% document

\usepackage{amsfonts}
\usepackage{amsmath}
\usepackage{amssymb}
\usepackage{amsthm}
\usepackage{color}
\usepackage{float}
\usepackage{graphicx}
\usepackage{hyperref}
\usepackage{times}
\usepackage{url}
\usepackage{xcolor}

% document

\begin{document}

\section*{The Ratio of Area to Circumference is Half the Radius \begin{small} \begin{color}{gray} \\
(December 26, 2025) \end{color} \end{small}}

Pretending that we know some integral calculus but not how to measure circles, we know that the area of the first quadrant of a circle with radius $r$ centered at the origin is 
\[
 A = \int_0^r \sqrt{r^2-x^2}\, dx
\]
and its length is 
\[
 L = \int_0^r \sqrt{ 1 + \lp\frac{d}{dx}\lp \sqrt{r^2-x^2} \rp\rp^2}\, dx = \int_0^r \frac{r}{\sqrt{r^2-x^2}}\, dx
\]
This theorem connects these two integrals:

\begin{thm} For $\ds n=2,3,\ldots,$ and $\ds r>0$,
\begin{equation} \label{IntegralResult}
  \int_0^r \frac{r^n}{\sqrt{r^n-x^n}} \, dx = \lp 1+ \frac{2}{n} \rp \int_0^r \sqrt{r^n-x^n}\, dx
\end{equation}
\end{thm}

\begin{proof} Observing that 
\begin{align*}
  \int_0^r \sqrt{r^n-x^n}\, dx &= \int_0^r \frac{r^n-x^n}{\sqrt{r^n-x^n}}\, dx \\
  &= \int_0^r \frac{r^n}{\sqrt{r^n-x^n}}\, dx - \int_0^r \frac{x^n}{\sqrt{r^n-x^n}}\, dx \\
  &= \int_0^r \frac{r^n}{\sqrt{1-x^n}}\, dx + \int_0^r \frac{2x}{n}\cdot \frac{d}{dx} \sqrt{r^n-x^n}\, dx
\end{align*}
Applying integration by parts to the second integral on the right with $\ds u = 2x/n$ and $\ds dv = \frac{d}{dx} \sqrt{r^n-x^n}\, dx$ gives
\begin{align*}
  \int_0^r \frac{2x}{n}\cdot \frac{d}{dx} \sqrt{r^n-x^n}\, dx &= \left[ \frac{2x}{n} \sqrt{r^n-x^n} \right]_0^r - \frac{2}{n} \int_0^r \sqrt{r^n-x^n}\, dx \\
  &= - \frac{2}{n} \int_0^r \sqrt{r^n-x^n}\, dx
\end{align*}
and (\ref{IntegralResult}) follows.
\end{proof}

Taking $n=2$ gives 
\[
  r L = \lp 1+ \frac{2}{2} \rp A \textup{ or } \frac{A}{L} = \frac{r}{2}
\]

\end{document}

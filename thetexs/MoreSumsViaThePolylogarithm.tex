% preamble

\documentclass[12pt]{article}

% preamble

% TO DOUBLESPACE THE PRINTOUT, INSERT THE COMMAND
% \renewcommand{\baselinestretch}{2}

%\setlength{\textheight}{8.5in}
%\setlength{\textwidth}{6.25in}
%\setlength{\topmargin}{0.0in}

\newtheorem{defn}{Definition}
\newtheorem{cor}[defn]{Corollary}
\newtheorem{lemma}[defn]{Lemma}
\newtheorem{obs}[defn]{Observation}
\newtheorem{prop}[defn]{Proposition}
\newtheorem{thm}[defn]{Theorem}
\newtheorem{cond}[defn]{Condition}
\newtheorem{conj}[defn]{Conjecture}
\newtheorem{ass}[defn]{Assumption}
\newtheorem{example}[defn]{Example}
\newtheorem{rem}[defn]{Remark}

\newcommand{\abs}[1]{\left| #1 \right| }
\newcommand{\ans}{\noi\textbf{Answer: }}
\newcommand{\ds}{\displaystyle}
\newcommand{\dydx}{\ds \frac{dy}{dx}}
\newcommand{\infnorm}[1]{\ensuremath{\left\| #1 \right\|_{\infty}}}
\newcommand{\ital}{\textit}
\newcommand{\la}{\langle}
\newcommand{\lb}{\left\{}
\newcommand{\li}{\mathrm{Li}}
\newcommand{\limn}{\lim_{n\rightarrow\infty}}
\newcommand{\lp}{\left(}
\newcommand{\Mod}[1]{\ (\mathrm{mod}\ #1)}
\newcommand{\N}{I\!\!N}
\newcommand{\noi}{\noindent}
\newcommand{\norm}[1]{\ensuremath{\left\| #1 \right\| }}
\newcommand{\oon}{\frac{1}{n}}
\newcommand{\pic}[1]{\begin{center}\includegraphics{#1}\end{center}}
\newcommand{\R}{I\!\!R}
\newcommand{\ra}{\rangle}
\newcommand{\rb}{\right\}}
\newcommand{\rp}{\right)}
\newcommand{\skp}{\vspace{\baselineskip}}
\newcommand{\snsp}{@!@!@!@!@!}
\newcommand{\trm}{\textrm}
\newcommand{\ve}{\ensuremath{\varepsilon}}

% document

\usepackage{amsfonts}
\usepackage{amsmath}
\usepackage{amssymb}
\usepackage{amsthm}
\usepackage{color}
\usepackage{float}
\usepackage{graphicx}
\usepackage{hyperref}
\usepackage{times}
\usepackage{url}
\usepackage{xcolor}

% document

\begin{document}

\section*{More Sums via the Polylogarithm \begin{small} \begin{color}{gray} (July 15, 2025) \end{color} \end{small}}

Differentiating $\ds \li_0(z)$ twice, we get
\[
  \li_0''(z) = \frac{d}{dz} \left[ \frac{1}{z} \li_{-1}(z) \right]
   = -\frac{1}{z^2} \li_{-1}(z) + \frac{1}{z^2} \li_{-2}(z)
\]
Because we have closed form formulas for $\ds \li_0(z)$ and $\ds \li_{-1}(z)$, this allows us to compute such a formula for $\ds \li_{-2}(z)$, which is
\begin{equation} \label{Li2}
  \li_{-2}(z) = \sum_{n=1}^{\infty} n^2 z^n = -\frac{z(z+1)}{(z-1)^3}
\end{equation}
And again, taking $\ds z=1/2, 1/3, 1/4$, we can compute some remarkable sums.
\begin{eqnarray*}
  \sum_{n=1}^{\infty} \frac{n^2}{2^n} & = & 6 \\
  \sum_{n=1}^{\infty} \frac{n^2}{3^n} & = & \frac{3}{2} \\
  \sum_{n=1}^{\infty} \frac{n^2}{4^n} & = & \frac{20}{27} \\
\end{eqnarray*}

Again, it is the case that if $z = a/(a+1)$, then substituting into the right side of (\ref{Li2}) gives 
\[
  \sum_{n=1}^{\infty} n^2 \lp \frac{a}{a+1}\rp^n = a(a+1)(2a+1)
\]
and again we get some sums with perhaps unexpected integer values
\[
  \sum_{n=1}^{\infty} n^2 \lp \frac{2}{3} \rp^n = 30, \sum_{n=1}^{\infty} n^2 \lp \frac{6}{7} \rp^n= 546, \textup{ etc.}
\]

\end{document}

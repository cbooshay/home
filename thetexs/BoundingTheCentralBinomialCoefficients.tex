\documentclass[12pt]{article}

% preamble

% TO DOUBLESPACE THE PRINTOUT, INSERT THE COMMAND
% \renewcommand{\baselinestretch}{2}

%\setlength{\textheight}{8.5in}
%\setlength{\textwidth}{6.25in}
%\setlength{\topmargin}{0.0in}

\newtheorem{defn}{Definition}
\newtheorem{cor}[defn]{Corollary}
\newtheorem{lemma}[defn]{Lemma}
\newtheorem{obs}[defn]{Observation}
\newtheorem{prop}[defn]{Proposition}
\newtheorem{thm}[defn]{Theorem}
\newtheorem{cond}[defn]{Condition}
\newtheorem{conj}[defn]{Conjecture}
\newtheorem{ass}[defn]{Assumption}
\newtheorem{example}[defn]{Example}
\newtheorem{rem}[defn]{Remark}

\newcommand{\abs}[1]{\left| #1 \right| }
\newcommand{\ans}{\noi\textbf{Answer: }}
\newcommand{\ds}{\displaystyle}
\newcommand{\dydx}{\ds \frac{dy}{dx}}
\newcommand{\infnorm}[1]{\ensuremath{\left\| #1 \right\|_{\infty}}}
\newcommand{\ital}{\textit}
\newcommand{\la}{\langle}
\newcommand{\lb}{\left\{}
\newcommand{\lp}{\left(}
\newcommand{\N}{I\!\!N}
\newcommand{\noi}{\noindent}
\newcommand{\norm}[1]{\ensuremath{\left\| #1 \right\| }}
\newcommand{\oon}{\frac{1}{n}}
\newcommand{\pic}[1]{\begin{center}\includegraphics{#1}\end{center}}
\newcommand{\R}{I\!\!R}
\newcommand{\ra}{\rangle}
\newcommand{\rb}{\right\}}
\newcommand{\rp}{\right)}
\newcommand{\skp}{\vspace{\baselineskip}}
\newcommand{\snsp}{@!@!@!@!@!}
\newcommand{\trm}{\textrm}
\newcommand{\ve}{\ensuremath{\varepsilon}}

% document

\usepackage{amsmath}
\usepackage{graphicx} 
\usepackage{hyperref}
\usepackage{soul}
\usepackage{xcolor}

% document

\begin{document}

%. John D. Cook's blog post of 13 July 2018

\section*{Bounds for the Central Binomial Coefficients \\
\begin{small} \begin{color}{gray} (January 9, 2025) \end{color} \end{small}}

Just how big are the central binomial coefficients? A plot of $\ds \log(\gamma_n)$ looks linear with slope $\ds \approx 1.4$ suggesting $\ds \gamma_n \approx e^{1.4n}$.

\pic{images/LogGamman}

An old problem in \textit{The College Mathematics Journal} \cite{WANGProblem, WANGSolution} is to establish the bounds
\begin{equation} \label{wang}
  \frac{2^{2n-1}}{\sqrt{n}} < \gamma_n < \frac{2^{2n-1/2}}{\sqrt{n}}
\end{equation}
or, equivalently, 
\[
  \frac{4^n}{2\sqrt{n}} < \gamma_n < \frac{4^n}{\sqrt{2 n}}
\]
An elegant proof of (\ref{wang}) is given by Henry O. Pollak, but a tighter upper bound can be given beginning by expanding $\ds \cos^{2n}(x)$ using the definition $\ds \cos(x) = (e^{ix} + e^{-ix})/2$ and the Binomial Theorem.
\begin{equation} \label{CosineExpansion}
  \cos^{2n}(x) = \lp \frac{e^{ix} + e^{-ix}}{2} \rp^{2n} % = \frac{1}{4^n} \sum_{k=0}^{2n} \binom{2n}{k} e^{i k x} e^{-i(2n-k)x}
  = \frac{1}{4^n} \sum_{k=0}^{2n} \binom{2n}{k} e^{2i(k-n)x} 
\end{equation}
If $m$ is an even, nonzero integer,
\[
  \int_{-\pi/2}^{\pi/2} e^{i m x}\, dx = \frac{1}{i m} \lp e^{i m (\pi/2)} - e^{-i m(\pi/2)} \rp = 0
\]
And of course, if $m=0$, then
\[
  \int_{-\pi/2}^{\pi/2} e^{i m x}\, dx = \int_{-\pi/2}^{\pi/2} (1)\, dx = \pi
\]
So if both sides of (\ref{CosineExpansion}) are integrated over $\ds [-\pi/2, \pi/2]$, the integral of every term in the sum on the right is zero except for the $\ds k=n$ term, giving
\begin{equation} \label{Cos2n}
  \int_{-\pi/2}^{\pi/2} \cos^{2n}(x)\, dx = \frac{\pi}{4^n}\binom{2n}{n}
\end{equation} 

One can bound $\ds \gamma_n$, then, by bounding the cosine function on the interval $\ds [-\pi/2, \pi/2]$, and an excellent such bound is $\ds \cos(x) \leq e^{-x^2/2}$. 

\begin{align*}
 \gamma_n &\leq \frac{4^n}{\pi} \int_{-\pi/2}^{\pi/2} \cos^{2n}(x)\, dx \\
 &\leq\frac{4^n}{\pi} \int_{-\pi/2}^{\pi/2} e^{-nx^2}\, dx \\
 &< \frac{4^n}{\pi} \int_{-\infty}^{\infty} e^{\frac{-x^2}{2\cdot \frac{1}{2n}}}\, dx \\
 &= \frac{4^n}{\pi} \sqrt{2 \pi} \sqrt{\frac{1}{2n}} \\
 &= \frac{4^n}{\sqrt{n \pi}}
\end{align*}
The integrand in the third line is a constant multiple of the density function of an $\ds N \lp0, \sqrt{1/(2n)}\rp$ distribution.

\begin{thebibliography}{1}

\bibitem{WANGProblem}  Problem 420 proposed by Edward T.H. Wang. \textit{The College Mathematics Journal}, Vol. 21, No. 1.

\bibitem{WANGSolution} Pollak, Henry O. and others. Solution to problem 420. \textit{The College Mathematics Journal}, Vol. 22, No. 1.

\end{thebibliography}{1}

\end{document}

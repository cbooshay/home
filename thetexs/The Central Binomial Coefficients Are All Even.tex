% preamble

\documentclass[12pt]{article}

% preamble

% TO DOUBLESPACE THE PRINTOUT, INSERT THE COMMAND
% \renewcommand{\baselinestretch}{2}

%\setlength{\textheight}{8.5in}
%\setlength{\textwidth}{6.25in}
%\setlength{\topmargin}{0.0in}

\newtheorem{defn}{Definition}
\newtheorem{cor}[defn]{Corollary}
\newtheorem{lemma}[defn]{Lemma}
\newtheorem{obs}[defn]{Observation}
\newtheorem{prop}[defn]{Proposition}
\newtheorem{thm}[defn]{Theorem}
\newtheorem{cond}[defn]{Condition}
\newtheorem{conj}[defn]{Conjecture}
\newtheorem{ass}[defn]{Assumption}
\newtheorem{example}[defn]{Example}
\newtheorem{rem}[defn]{Remark}

\newcommand{\abs}[1]{\left| #1 \right| }
\newcommand{\ans}{\noi\textbf{Answer: }}
\newcommand{\ds}{\displaystyle}
\newcommand{\dydx}{\ds \frac{dy}{dx}}
\newcommand{\infnorm}[1]{\ensuremath{\left\| #1 \right\|_{\infty}}}
\newcommand{\ital}{\textit}
\newcommand{\la}{\langle}
\newcommand{\lb}{\left\{}
\newcommand{\lp}{\left(}
\newcommand{\N}{I\!\!N}
\newcommand{\noi}{\noindent}
\newcommand{\norm}[1]{\ensuremath{\left\| #1 \right\| }}
\newcommand{\oon}{\frac{1}{n}}
\newcommand{\pic}[1]{\begin{center}\includegraphics{#1}\end{center}}
\newcommand{\R}{I\!\!R}
\newcommand{\ra}{\rangle}
\newcommand{\rb}{\right\}}
\newcommand{\rp}{\right)}
\newcommand{\skp}{\vspace{\baselineskip}}
\newcommand{\snsp}{@!@!@!@!@!}
\newcommand{\trm}{\textrm}
\newcommand{\ve}{\ensuremath{\varepsilon}}

% document

\usepackage{amsmath}
\usepackage{graphicx} 
\usepackage{hyperref}
\usepackage{soul}
\usepackage{xcolor}

% document

\begin{document}

\section*{The Central Binomial Coefficients Are All Even \begin{small} \begin{color}{gray} \\
(\today) \end{color} \end{small}}

This is the latest in a series of posts about the prime factorizations of the Central Binomial Coefficients, the numbers $\ds \gamma_n  = \binom{2n}{n}$. So far, we have noted that if a prime $p$ is such that either $\ds 2n/3<p\leq n$ or $\ds p>2n$, then $p$ does not appear in the prime factorization of $\ds \gamma_n$, and if $\ds n<p\leq 2n$, then $p$ appears to the first power in the prime factorization of $\ds \gamma_n$. This leaves the contribution of the primes between 2 and $\ds 2n/3$ undetermined. A tiny step forward can be taken using very basic properties of Pascal's Triangle.

The binomial coefficient $\ds \binom{r}{k}$ is the number in row $r$ and column $k$ in Pascal's famous triangle (in the cases of both rows and columns, the count begins at 0). The rows are symmetric, that is $\ds \binom{r}{k} = \binom{r}{r-k}$, which means that $\ds \binom{2n-1}{n-1} = \binom{2n-1}{n}$. Also, every number in the triangle is the sum of the two numbers above it, a fact known as Pascal's Identity. Applying this to the central binomial coefficients gives:
\[
  \binom{2n}{n} = \binom{2n-1}{n-1}+\binom{2n-1}{n} = (2) \binom{2n-1}{n}
\]
This means $\ds \gamma_n$ is always even, so $2$ \textit{always} appears in its prime factorization, and the question becomes what power of $2$ appears in this prime factorization.

The largest power of a prime $p$ that divides an integer $n$ is called the $p-$adic valuation of $n$ and symbolized $\ds v_p(n)$. Below is a graph of $\ds v_2(\gamma_n)$ for the first 100 values of $n$

\pic{images/TwoValuation}

The function is not periodic but does have a repetitive structure.

\end{document}
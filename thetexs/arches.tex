% \documentclass[12pt]{article}

\documentclass[12pt]{article}

% preamble

% TO DOUBLESPACE THE PRINTOUT, INSERT THE COMMAND
% \renewcommand{\baselinestretch}{2}

%\setlength{\textheight}{8.5in}
%\setlength{\textwidth}{6.25in}
%\setlength{\topmargin}{0.0in}

\newtheorem{defn}{Definition}
\newtheorem{cor}[defn]{Corollary}
\newtheorem{lemma}[defn]{Lemma}
\newtheorem{obs}[defn]{Observation}
\newtheorem{prop}[defn]{Proposition}
\newtheorem{thm}[defn]{Theorem}
\newtheorem{cond}[defn]{Condition}
\newtheorem{conj}[defn]{Conjecture}
\newtheorem{ass}[defn]{Assumption}
\newtheorem{example}[defn]{Example}
\newtheorem{rem}[defn]{Remark}

\newcommand{\abs}[1]{\left| #1 \right| }
\newcommand{\ans}{\noi\textbf{Answer: }}
\newcommand{\ds}{\displaystyle}
\newcommand{\dydx}{\ds \frac{dy}{dx}}
\newcommand{\infnorm}[1]{\ensuremath{\left\| #1 \right\|_{\infty}}}
\newcommand{\ital}{\textit}
\newcommand{\la}{\langle}
\newcommand{\lb}{\left\{}
\newcommand{\lp}{\left(}
\newcommand{\N}{I\!\!N}
\newcommand{\noi}{\noindent}
\newcommand{\norm}[1]{\ensuremath{\left\| #1 \right\| }}
\newcommand{\oon}{\frac{1}{n}}
\newcommand{\pic}[1]{\begin{center}\includegraphics{#1}\end{center}}
\newcommand{\R}{I\!\!R}
\newcommand{\ra}{\rangle}
\newcommand{\rb}{\right\}}
\newcommand{\rp}{\right)}
\newcommand{\skp}{\vspace{\baselineskip}}
\newcommand{\snsp}{@!@!@!@!@!}
\newcommand{\trm}{\textrm}
\newcommand{\ve}{\ensuremath{\varepsilon}}

% document

\usepackage{amsmath}
\usepackage{graphicx} 
\usepackage{hyperref}
\usepackage{soul}
\usepackage{xcolor}

% document

\begin{document}

\section*{Arches \begin{small} \begin{color}{gray} \\
(\today) \end{color} \end{small}}

As someone who has taught more calculus than most, my mind spends a lot of time in the plane thinking about curves there, and I have always been fascinated by the subtle (and sometimes not-so-subtle) differences among them. Take the connection of one point on the $x-$axis to another by an ``arch." Without being overly rigorous, let's say that an arch is a curve that lies on or above a secant line over any subinterval--we're aiming at concave down--and achieves a maximum value at the midpoint of the two $x-$values.

We may as well position the points at $\ds (-r,0), (0,r)$ for some $r>0$ that is half the distance between the two original points, so our arch will pass through the points $\ds (\pm r,0)$ and $\ds (0,r)$. Any arch will sit in the envelope between the triangle with these vertices and the rectangle $\ds [-r,r]\times [0,r]$. Surely, the curved arch that jumps to mind first is a circular arch, i.e., the graph of the function $\ds y = \sqrt{r^2-x^2}$. Measuring the area under these arches gives
\[
\begin{array}{l l}
\text{triangle} & \frac{1}{2}(2r)(r) = r^2 \\
\text{circle} & \frac{\pi}{2} r^2 \\
% \text{cycloid} & \frac{3}{\pi} r^2 \\
\text{rectangle} & (2r)(r) = 2r^2
\end{array}
\]

A parabolic arch would have the form $\ds y = c(r-x)(r+x)$, and to pass through the point $\ds (0,r)$ requires $\ds c = \frac{1}{r}$, so $\ds y = \frac{r^2-x^2}{r}$. To make a sinusoidal arch, we shift the sine function $r$ units left via $\ds y = \sin(x+r)$, scale it by a factor of $r$ to put the peak at the right height (so now $\ds y = r \sin(x+r)$). Finally, we would like to stretch the graph horizontally to make the point $\ds x=r$ on our sinusoidal arch feel like the point $\ds x=\pi$ on the sine function. Setting $\ds c(r+r) = \pi$ gives $\ds c = \pi/2r$, and so our sinusoidal arch is $\ds \lp r \sin \lp \frac{\pi(x+r)}{2r} \rp \rp$.

A little integral calculus shows how these arches compare to the others.
\[
  \int_{-r}^r \frac{r^2-x^2}{r}\, dx = \frac{4}{3} r^2 \text{ and }
  \int_{-r}^r r \sin \lp \frac{\pi(x+r)}{2r} \rp\, dx = \frac{4}{\pi} r^2
\]
So,
\begin{eqnarray*}
\text{triangle} & r^2 \\
\text{sinusoidal} & \frac{4}{\pi} r^2 \\
\text{parabolic} & \frac{4}{3} r^2 \\
\text{circular} & \frac{\pi}{2} r^2 \\
\text{rectangle} & 2r^2
\end{eqnarray*}

The parabolic arch can be generalized to a polynomial arch $\ds y = \frac{r^n-x^n}{r^{n-1}}$ for even $n$. This arch encloses an area of $\ds \frac{2n}{n+1} r^2$ and fills out the outer rectangle as $n$ grows.

\end{document}

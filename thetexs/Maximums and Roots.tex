% preamble

\documentclass[12pt]{article}

% preamble

% TO DOUBLESPACE THE PRINTOUT, INSERT THE COMMAND
% \renewcommand{\baselinestretch}{2}

%\setlength{\textheight}{8.5in}
%\setlength{\textwidth}{6.25in}
%\setlength{\topmargin}{0.0in}

\newtheorem{defn}{Definition}
\newtheorem{cor}[defn]{Corollary}
\newtheorem{lemma}[defn]{Lemma}
\newtheorem{obs}[defn]{Observation}
\newtheorem{prop}[defn]{Proposition}
\newtheorem{thm}[defn]{Theorem}
\newtheorem{cond}[defn]{Condition}
\newtheorem{conj}[defn]{Conjecture}
\newtheorem{ass}[defn]{Assumption}
\newtheorem{example}[defn]{Example}
\newtheorem{rem}[defn]{Remark}

\newcommand{\abs}[1]{\left| #1 \right| }
\newcommand{\ans}{\noi\textbf{Answer: }}
\newcommand{\ds}{\displaystyle}
\newcommand{\dydx}{\ds \frac{dy}{dx}}
\newcommand{\infnorm}[1]{\ensuremath{\left\| #1 \right\|_{\infty}}}
\newcommand{\ital}{\textit}
\newcommand{\la}{\langle}
\newcommand{\lb}{\left\{}
\newcommand{\li}{\mathrm{Li}}
\newcommand{\limn}{\lim_{n\rightarrow\infty}}
\newcommand{\lp}{\left(}
\newcommand{\Mod}[1]{\ (\mathrm{mod}\ #1)}
\newcommand{\N}{I\!\!N}
\newcommand{\noi}{\noindent}
\newcommand{\norm}[1]{\ensuremath{\left\| #1 \right\| }}
\newcommand{\oon}{\frac{1}{n}}
\newcommand{\pic}[1]{\begin{center}\includegraphics{#1}\end{center}}
\newcommand{\R}{I\!\!R}
\newcommand{\ra}{\rangle}
\newcommand{\rb}{\right\}}
\newcommand{\rp}{\right)}
\newcommand{\skp}{\vspace{\baselineskip}}
\newcommand{\snsp}{@!@!@!@!@!}
\newcommand{\trm}{\textrm}
\newcommand{\ve}{\ensuremath{\varepsilon}}

% document

\usepackage{amsfonts}
\usepackage{amsmath}
\usepackage{amssymb}
\usepackage{amsthm}
\usepackage{color}
\usepackage{float}
\usepackage{graphicx}
\usepackage{hyperref}
\usepackage{times}
\usepackage{url}
\usepackage{xcolor}

% document

\begin{document}

\section*{Roots and Maximums of Random Numbers \begin{small} \begin{color}{gray} (November 24, 2025) \end{color} \end{small}}

Matt Parker, proprietor of the YouTube channel ``Stand-up Maths," does a nice video explaining why the $\ds n^{th}$ root of a random number (from the unit interval) has the same distribution as the maximum of $n$ independent such numbers. That is, if 
\begin{align*}
  Y_1 &= \sqrt[n]{X} \\
  Y_2 &= \max (X_1, X_2, \ldots, X_n)
\end{align*}
where all the $X$'s are drawn randomly from $\ds [0,1]$, then $\ds Y_1$ and $\ds Y_2$ have exactly the same probability distribution, which may or may not be counterintuitive. 

Showing this is a nice exercise for undergraduate probability students as the tools of calculus give the result easily and elegantly. What we mean by ``$X$ is drawn randomly from $\ds [0,1]$" is that $X$ has probability density function $\ds f(x) = 1$ and cumulative distribution function $\ds F(x) = x$ (for $\ds x\in [0,1]$). If we know that the pdf is the derivative of the cdf, and we know differential calculus, then
\begin{align*}
 f_{Y_1}(y) &= \frac{d}{dy} P\lp \sqrt[n]{X} \leq y \rp \\
 &= \frac{d}{dy} F(y^n) \\
 &= f(y^n) n y^{n-1} \\
 &= n y^{n-1}
\end{align*}
Also,
\begin{align*}
 f_{Y_2}(y) &= \frac{d}{dy} P\lp \max (X_1, X_2, \ldots, X_n) \leq y \rp \\
 &= \frac{d}{dy} P(X_1\leq y)P(X_2\leq y) \cdots P(X_n\leq y) \\
 &= \frac{d}{dy} F(y)^n \\
 &= n F(y)^{n-1} f(y) \\
 &= n y^{n-1}
\end{align*}

\end{document}

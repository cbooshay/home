% preamble

\documentclass[12pt]{article}

% preamble

% TO DOUBLESPACE THE PRINTOUT, INSERT THE COMMAND
% \renewcommand{\baselinestretch}{2}

%\setlength{\textheight}{8.5in}
%\setlength{\textwidth}{6.25in}
%\setlength{\topmargin}{0.0in}

\newtheorem{defn}{Definition}
\newtheorem{cor}[defn]{Corollary}
\newtheorem{lemma}[defn]{Lemma}
\newtheorem{obs}[defn]{Observation}
\newtheorem{prop}[defn]{Proposition}
\newtheorem{thm}[defn]{Theorem}
\newtheorem{cond}[defn]{Condition}
\newtheorem{conj}[defn]{Conjecture}
\newtheorem{ass}[defn]{Assumption}
\newtheorem{example}[defn]{Example}
\newtheorem{rem}[defn]{Remark}

\newcommand{\abs}[1]{\left| #1 \right| }
\newcommand{\ans}{\noi\textbf{Answer: }}
\newcommand{\ds}{\displaystyle}
\newcommand{\dydx}{\ds \frac{dy}{dx}}
\newcommand{\infnorm}[1]{\ensuremath{\left\| #1 \right\|_{\infty}}}
\newcommand{\ital}{\textit}
\newcommand{\la}{\langle}
\newcommand{\lb}{\left\{}
\newcommand{\li}{\mathrm{Li}}
\newcommand{\limn}{\lim_{n\rightarrow\infty}}
\newcommand{\lp}{\left(}
\newcommand{\Mod}[1]{\ (\mathrm{mod}\ #1)}
\newcommand{\N}{I\!\!N}
\newcommand{\noi}{\noindent}
\newcommand{\norm}[1]{\ensuremath{\left\| #1 \right\| }}
\newcommand{\oon}{\frac{1}{n}}
\newcommand{\pic}[1]{\begin{center}\includegraphics{#1}\end{center}}
\newcommand{\R}{I\!\!R}
\newcommand{\ra}{\rangle}
\newcommand{\rb}{\right\}}
\newcommand{\rp}{\right)}
\newcommand{\skp}{\vspace{\baselineskip}}
\newcommand{\snsp}{@!@!@!@!@!}
\newcommand{\trm}{\textrm}
\newcommand{\ve}{\ensuremath{\varepsilon}}

% document

\usepackage{amsfonts}
\usepackage{amsmath}
\usepackage{amssymb}
\usepackage{amsthm}
\usepackage{color}
\usepackage{float}
\usepackage{graphicx}
\usepackage{hyperref}
\usepackage{times}
\usepackage{url}
\usepackage{xcolor}

% document

\begin{document}

\section*{The Central Binomial Coefficients (1) \begin{small} \begin{color}{gray} \\
(\today) \end{color} \end{small}}

The numbers $\ds \binom{2n}{n}$ that run vertically down the center of Pascal's Triangle are called the \textit{central binomial coefficients}. I'll use the notation $\ds \gamma_n \doteq \binom{2n}{n}$ for these numbers, which grow quickly with $n$

\[
\begin{array}{l | l}
n & \gamma_n \\ \hline
1 & 2 \\
5 & 252 \\
10 & 184756 \\
15 & 155117520
\end{array}
\]
These numbers are interesting for a number of reasons, one of which is their prime factorizations. A key feature of primes is that if a prime number divides a product, then it must divide one of the factors, so for example the prime 3 divides 36, and no matter how you factor 36, 3 will always divide at least one of the factors.
\[
\begin{array}{c}
  (1) \begin{color}{red}{(36)} \end{color} \\
  (2) \begin{color}{red}{(18)} \end{color} \\
  \begin{color}{red} (3) (12) \end{color} \\
  (4) \begin{color}{red}{(9)} \end{color} \\
  \begin{color}{red} (6) (6) \end{color} \\
\end{array}
\]
This is not the case for non-primes. The number 6 divides 36, but in the facorization $\ds 36 = 4\cdot 9$, 6 divides neither factor.

Any prime that divides $\ds \gamma_n = \frac{(2n)!}{(n!)^2}$ must divide the numerator, and any prime that divides $\ds (2n)! = (2n)(2n-1)\cdots 2\cdot 1$ must divide one of the factors and so cannot be larger than $\ds 2n$. This means that while $\ds \gamma_n$ is a large number even for relatively small $n$, it can have no ``large" prime factors. For example, $\ds \gamma_{50} \approx 10^{29}$ but has no prime factor larger than 100.

Is this unusual? I had my computer randomly select 200 integers between $\ds 10^{24}$ and $\ds 10^{34}$, and the \textit{smallest} maximum prime factor of the bunch was $\ds 113578811$. Hmmm...

\end{document}

% preamble

\documentclass[12pt]{article}

% preamble

% TO DOUBLESPACE THE PRINTOUT, INSERT THE COMMAND
% \renewcommand{\baselinestretch}{2}

%\setlength{\textheight}{8.5in}
%\setlength{\textwidth}{6.25in}
%\setlength{\topmargin}{0.0in}

\newtheorem{defn}{Definition}
\newtheorem{cor}[defn]{Corollary}
\newtheorem{lemma}[defn]{Lemma}
\newtheorem{obs}[defn]{Observation}
\newtheorem{prop}[defn]{Proposition}
\newtheorem{thm}[defn]{Theorem}
\newtheorem{cond}[defn]{Condition}
\newtheorem{conj}[defn]{Conjecture}
\newtheorem{ass}[defn]{Assumption}
\newtheorem{example}[defn]{Example}
\newtheorem{rem}[defn]{Remark}

\newcommand{\abs}[1]{\left| #1 \right| }
\newcommand{\ans}{\noi\textbf{Answer: }}
\newcommand{\ds}{\displaystyle}
\newcommand{\dydx}{\ds \frac{dy}{dx}}
\newcommand{\infnorm}[1]{\ensuremath{\left\| #1 \right\|_{\infty}}}
\newcommand{\ital}{\textit}
\newcommand{\la}{\langle}
\newcommand{\lb}{\left\{}
\newcommand{\lp}{\left(}
\newcommand{\N}{I\!\!N}
\newcommand{\noi}{\noindent}
\newcommand{\norm}[1]{\ensuremath{\left\| #1 \right\| }}
\newcommand{\oon}{\frac{1}{n}}
\newcommand{\pic}[1]{\begin{center}\includegraphics{#1}\end{center}}
\newcommand{\R}{I\!\!R}
\newcommand{\ra}{\rangle}
\newcommand{\rb}{\right\}}
\newcommand{\rp}{\right)}
\newcommand{\skp}{\vspace{\baselineskip}}
\newcommand{\snsp}{@!@!@!@!@!}
\newcommand{\trm}{\textrm}
\newcommand{\ve}{\ensuremath{\varepsilon}}

% document

\usepackage{amsmath}
\usepackage{graphicx} 
\usepackage{hyperref}
\usepackage{soul}
\usepackage{xcolor}

% document

\begin{document}

\section*{The Central Binomial Coefficients: Prime Factors between $\ds 2n/3$ and $\ds 2n$ \begin{small} \begin{color}{gray} \\
(September 12, 2024) \end{color} \end{small}}

In my last \href{https://cbooshay.github.io/home/thepdfs/CBCNoLargePrimeFactors.pdf}{post}, I made the small observation that the central binomial coefficient
\begin{equation} \label{gamman}
  \gamma_n \doteq \binom{2n}{n} = \frac{(2n)!}{(n!)^2}
\end{equation}
has no prime factors larger than $\ds 2n$, which is a bit unusual for a number so large. So all the prime factors of $\ds \gamma_n$ are to be found in $\ds \lb 1, 2, \ldots, 2n \rb$, and most of the primes in this range occur in the prime factorization of $\ds \gamma_n$ either to the first power or not at all.

Every prime in the range $\ds \lb n+1, n+2, \ldots, 2n \rb$ divides $\ds (2n)!$ but is too large to divide $n!$, and so must appear in the prime factorization of $\ds \gamma_n$. However, if $p^2$ divides $\ds (2n)!$, then there must be a multiple of $p$ besides $p$ itself in range $\ds \lb 1, 2\ldots, 2n\rb$. For $\ds p>n$, this is impossible, so every prime in the range $\ds \lb n+1, n+2, \ldots, 2n \rb$ appears in the prime factorization of $\ds \gamma_n$ to the first power.

If $\ds \frac{2}{3}n <p\leq n$, then there is exactly one multiple of $p$ in the range $\ds \lb 1,2,\ldots,n\rb$ and exactly two multiples of $p$ in the range $\ds \lb 1,2,\ldots, 2n\rb$, so that in the fraction $\ds \frac{(2n)!}{(n!)^2}$, all the $p$'s will cancel and no such prime occurs in the prime factorization of $\ds \gamma_n$.

For example, if $n=50$, then the primes $\ds 53, 59, 61, 67, 71, 73, 79, 83, 89, 97$ (the primes between 51 and $100$) each occur to the first power in the prime factorization of $\ds \gamma_n$, so 
\begin{equation} \label{FactoringGamma50}
  \gamma_{50} = C (53)(59)(61)(67)(71)(73)(79)(83)(89)(97)
\end{equation}
Also, the primes $\ds 37, 41, 43, 47$ (the primes between $\ds 100/3$ and $\ds 50$) do not divide $\ds \gamma_{50}$, so the prime factors of $C$ are to be found among $\ds 2, 3, 5, 7, 11, 13, 17, 19, $ $23, 29, 31$. 
\end{document}
